
\subsection{What is in the Astronomer Software Tool Stack?}

In this section we consider the most common software tools for professional astronomers. We refer to the set of tools as "stack". In the survey form we suggested 19 software tools and allowed participants to add any options we missed. The input was edited to standardize spelling and capitalization of tools. In total, participants added 64 custom options. 10 respondents did not provide an answer to this question. While "C" was an option, "C++" was not part of our suggestions. Many participants noted in the comments what they chose "C" even thought they actually use "C++". For this reason we consider C and C++ together in our analysis. Withing the top-20 most used software tools there are 4 items that were not on our original list: C++, Mathematica, gnuplot and awk.

The overall astronomer stack is surprisingly narrow (Figure \ref{fig:stack1}, first panel). Only 10 software tools are used by more than 10\% of the survey participants. These are (from most popular to least popular): Python, shell scripting, IDL, C/C++, Fortran, IRAF, spreadsheets, HTML/CSS, SQL and Supermongo. Across all participants the most common programing language is Python ($67\pm2\%$), followed by IDL ($44\pm2\%$), C/C++ ($37\pm2\%$) and Fortran ($28\pm2\%$). Shell scripting is the second most popular tool for astronomers ($47\pm2\%$). The software system IRAF (Image Reduction and Analysis Facility) is used frequently by $24\pm1\%$ and survey participants. 

Across the different career stages, we notice that more senior astronomers have a broader tool stack, i.e. they utilize a wider variety of tools in their research. Only 8 tools are used by more than 10\% of graduate students, 9 tools are used by more than 10\% of postdocs and 11 tools are used by more than 10\% of faculty and scientists. Python is the most populat tool at all career levels and it is most popular among junior researchers. Four out of five graduate students use Python ($80\pm5\%$), as do $70\pm5\%$ of postdocs and half of faculty and scientists ($53\pm4\%$). IDL, IRAF and compiled languages have a more uniform user base across all career levels. Some tools are unique to certain demographics. Graduate students have the highest fraction of Matlab users ($11\%$), while faculty and research scientists dominate HTML/CSS (21\%), Supermongo (16\%) and perl (16\%).

Software use depentds strongly on the research area as can be expected (Figure \ref{fig:stack2}). Observational astronomers have the highest fractions of IDL ($48\pm2\%$) and IRAF ($31\pm2\%$) users. Theoretical researchers have the highest fractions of compiled language users: C/C++ with $56\pm4\%$ and Fortran with $50\pm4\%$. 
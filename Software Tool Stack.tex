
\subsection{What is in the Astronomer Software Tool Stack?}

In this section we consider the most common software tools for professional astronomers. We refer to the set of tools as "stack". In the survey form we suggested 19 software tools and allowed participants to add any options we missed. The input was edited to standardize spelling and capitalization of tools. In total, participants added 64 custom options. 10 respondents did not provide an answer to this question. While "C" was an option, "C++" was not part of our suggestions. Many participants noted in the comments what they chose "C" even thought they actually use "C++". For this reason we consider C and C++ together in our analysis. Withing the top-20 most used software tools there are 4 items that were not on our original list: C++, Mathematica, gnuplot and awk.

The overall astronomer stack is surprisingly narrow (Figure \ref{fig:stack1}, first panel). Only 10 software tools are used by more than 10\% of the survey participants. These are (from most popular to least popular): Python, shell scripting, IDL, C/C++, Fortran, IRAF, spreadsheets, HTML/CSS, SQL and Supermongo. Across all participants the most common programing language is Python ($67\pm2\%$), followed by IDL ($44\pm2\%$), C/C++ ($37\pm2\%$) and Fortran ($28\pm2\%$). Shell scripting is the second most popular tool for astronomers ($47\pm2\%$). The software system IRAF (Image Reduction and Analysis Facility) is used frequently by $24\pm1\%$ and survey participants. 

Across the different career stages, we notice that more senior astronomers have wider tool stack. 
\section{Introduction}

Undoubtedly, nowadays software is crucial for astronomical research. Digital images and numerical simulations are the bread and butter of astronomers today and anyone who is actively involved in research has a wide variety of software tools in their toolbox. Yet, we know very little about the software habits of astronomers. Understanding the prevalence of use of software tools and the relative audience of different tools can be important for a number of reasons (ok, I can think of two besides idle curiosity).

In the first place, the astronomical community has, in the past, provided funding and support for tools intended for the wider community. Examples of this include the Goddard IDL library (funded by the NASA ADP program), IRAF (supported and developed by AURA) and STSDAS (supported and developed by STScI). As the field develops, new tools will be required and we need to be focusing our efforts on tools that will have the widest user base and the lowest barrier to utilization. As our work here shows, the rapid rise in the user-base of Python relative to R suggests that the tools in the former are likely to get many more users and contributers than the latter. 

More recently, there has been a broad discussion of the importance of data analysis and software development training in astronomy (e.g., the WGAA AAS special session "Steps Towards Better Curricula" which was standing room only). Although astronomy and astrophysics have gone digital long ago, the traditional training of astronomy and physics students rarely provides skills in data analysis and software development. Better information on the needs of researchers as well as the current availability (or lack there of) of training opportunities can be used to motivate and focus future efforts. 

Big data is coming, we need better skills: Berriman and Groom, DOI:10.1145/2043174.2043190

http://arxiv.org/abs/1212.1915
Finally, as we push for more researchers to publish their software in order to make research more reproducible and reuse more software, we need realize that one of the most common reasons for not sharing software is shame. Before we force researchers to make code public, we need to provide them with resources to learn how to write good code. 

With these goals in mind, and motivated by a number of discussions during the .Astronomy 6 conference, we created a survey to explore the software habits of astronomers. This survey was also inspired by a similar inquiry into the software use of researchers as universities in the UK carried out by the Software Sustainability Institute \href{(http://www.software.ac.uk/blog/2014-12-04-its-impossible-conduct-research-without-software-say-7-out-10-uk-researchers#comment-14813}{(Hettrick et al., 2014, hereafter the SSI survey)}. We adopt many of the questions of the SSI survey and discuss similarities and differences between their results and ours. The SSI survey defined software as:

“Software that is used to generate, process or analyse results that you intend to appear in a publication (either in a journal, conference paper, monograph, book or thesis). Research software can be anything from a few lines of code written by yourself, to a professionally developed software package. Software that does not generate, process or analyse results - such as word processing software, or the use of a web search - does not count as ‘research software’ for the purposes of this survey.”

The results from this survey may not be representative because we did not follow standard procedures for soliciting participation from a complete sample of community members (e.g, the members of the AAS). We do not know what was the audience the survey reached and what fraction of that audience chose to fill it out. Even though we believe the conclusions are likely valid, we hope to inspire a more formal survey in the use of software in our community.

Where are the data available. Where is the ipython notebook.
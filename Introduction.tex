\section{Introduction}

Much of modern Astronomy research depends on software. Digital images or numerical simulations are the bread and butter of most astronomers today and anyone who is actively involved in astronomy research (or related fields) has a variety of software techniques in their toolbox. Yet we know very little detail about the software habits of astronomers. This motivates efforts to gain a greater understanding the prevalence of software tools, the demographics of their users, and the general state of software training in astronomy.

The astronomical community has, in the past, provided funding and support to software tools intended for the wider community. Examples of this include the Goddard IDL library (funded by the NASA ADP program), IRAF (supported and developed by AURA) and STSDAS (supported and developed by STScI). As the field develops, new tools will be required and we need to be focusing our efforts on tools that will have the widest user base and the lowest barrier to utilization. For example, as our work here shows, the much larger astronomy user base of Python relative to R suggests that the tools in the former are likely to get many more users and contributers than the latter. 

More recently, there has been a broad discussion of the importance of data analysis and software development training in astronomy (e.g., the special session at the 225th AAS ``Astroinformatics and Astrostatistics in Astronomical Research Steps Towards Better Curricula'', which was standing room only). Although astronomy and astrophysics went digital long ago, the formal training of astronomy and physics students rarely involves software development or data-intensive analysis techniques. Such skills are increasingly critical in the era of unbiquitous ``Big data'' (e.g., \citealt{Berriman_2011}, or the \href{http://www.noao.edu/meetings/bigdata/}{2015 NOAO Big Data conference}). Better information on the needs of researchers as well as the current availability (or lack there of) of training opportunities can be used to motivate and focus future efforts improving this aspect of the astronomy curriculum. 

Recently,  the Software Sustainability Institute carried out an inquiry into the software use of researchers in the UK  (\citealt{f824cd98-b953-4c08-96c8-2533188bc4c4}, and see also \href{http://wl.figshare.com/articles/1243288/embed?show_title=1}{an associated presentation}). This survey provides useful context for software usage by researchers, as well as a useful definition of ``research software'':
\begin{quote}
Software that is used to generate, process or analyze results that you intend to appear in a publication (either in a journal, conference paper, monograph, book or thesis). Research software can be anything from a few lines of code written by yourself, to a professionally developed software package. Software that does not generate, process or analyze results - such as word processing software, or the use of a web search - does not count as ‘research software’ for the purposes of this survey.
\end{quote}
However, this survey was limited to researchers at UK institutions, and more importantly, was not limited to astronomers.  Hence, it inspired an effort to create a similar data set appropriate for the specific case of astronomy and related fields.




Motivated by these issues and related discussions during the .Astronomy 6 conference, we created a survey to explore the software habits of astronomers. 

Before moving on to discussing the \url{https://github.com/eteq/software_survey_analysis}
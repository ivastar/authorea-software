\section{Comparison to SSI Survey}
\label{sec:ssicompare}

This work was inspired by a UK survey led by the Software Sustainability Institute. While their findings are for the wider scientific community in the UK and ours are for the worldwide astronomical community, a comparison is still interesting. 

90\% of astronomers write some of software (93\% of UK astronomers). This is much larger than the SSI survey, which finds that only 56\% of researchers, across all disciplines, write some of their software.  This implies astronomers are much more dependent on their own software than other sciences.

In the SSI survey, 55\% of respondents say they have received some training in software development, with 40\% indicating that the training was a formal course and 15\% indicating self-directed study. Our categories are not identical, but we find that a similar fraction of astronomers - 57\% (53\% of UK astronomers) say that they have received some form of training. However, only 8\% say that the training was substantial, while 49\% say they received a little training. The decision what constitutes a lot and a little was left to the participants. Those who chose to expand on their decision indicated that a lot corresponded to a formal class while a little corresponded to using on-line materials such as Software Carpentry and Code Academy.

Finally, 42\% of our survey respondents who write their own software have received no training. This fraction is twice as large as the one reported in the SSI survey (21\%).  


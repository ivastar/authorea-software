\section{Data and Methods}

The survey was constructed as a Google questionnaire with seven questions and one comment box. Four of the questions were about software use and most were inspired by the UK survey:
\begin{enumerate}
\item Do you use software in your research?
\item Have you had formal training in software development?
\item Which of these is more common in your work? (on writing one's own software)
\item Select any of these that you use regularly to write code for your research. (on most commonly used software tools)
\end{enumerate}
Following the original survey, we also considered asking participants if they could no longer use research software, however we decided that such a question is unnecessary as all current astronomy research would be impossible without software. 
The remaining three questions requested basic demographic information:
\begin{enumerate}
\item What is your field of research?
\item What is your career stage?
\item What is the location of your institution?
\end{enumerate}

The survey was opened on December 9, 2014. The attendees of the .Astronomy conference were asked to forward it to their home departments. A link to the survey was also posted on the astronomy Facebook page. The survey received 758 responses in the first day (12/9) and another 210 during the second day (12/10). The data for this work was collected on February 3, 2015. The number of participants at that time was 1145. Three responses from participants who indicated that they work in fields other than astronomy were removed for a final tally of 1142 participants. 

The demographics of the sample are the following. Of the 1142 participants, 380 are graduate students, 340 are postdocs, 385 are research scientists and faculty (175 and 200, respectively). The remaining 37 are undergraduate students (10), emeritus professors, support scientists at observatories, adjunct faculty, post-bachelor's researchers, etc. These 37 "miscellaneous" career levels will be included in the analysis of the full sample but will not be included in any of the other career subgroups. For the analysis, we combine the research scientist and faculty subgroups to create three groups of similar size, roughly corresponding to "early", "intermediate" and "late" career stages.

In terms of their area of research, 823 participants chose "Observational Astronomy/Astrophysics", 353 selected "Theoretical Astronomy/Astrophysics", 130 indicated that they work in Astronomical Instrumentation and 66 in Planetary Science. 22 participants did not choose any of these four main categories. Of these, nine participants selected "Other", three did not choose and area of research, and 10 entered custom values such as physics, astro-statistics, cosmology, astroparticle physics, space physics, etc. Participants were allowed to choose more than one area of research, which is why the numbers for the individual categories add to more than 1142. 

The final piece of demographic data we collected was the geographic location of the participants' home institution. The majority of participants are from the USA (546), followed by Germany (170), UK (90), Australia (69) and Chile (35). 80\% of the participants come from these five top countries, with 48\% from the USA. The remaining 232 participants come from 41 different countries. The break down is the following:

             Netherlands     (31),
                  Sweden     (21),
               Argentina     (19),
                  Canada     (18),
                  Brazil     (14),
                  France     (13),
                   Spain     (12),
                   Italy     (11),
                  Poland      (9),
                  Mexico      (9),
             Switzerland      (9),
                  Israel      (6),
                 Denmark      (5),
                 Finland      (3),
                   India      (3),
                 Ireland      (3),
                Portugal      (3),
                   Japan      (3),
    United Arab Emirates      (2),
             South Korea      (2),
            South Africa      (2),
                  Russia      (2),
             Korea South      (2),
                 Belgium      (2),
                 Austria      (2),
             New Zealand      (2),
                  Greece      (1),
               Lithuania      (1),
                 Georgia      (1),
                Malaysia      (1),
                  Norway      (1),
                Slovakia      (1),
          Czech Republic      (1),
                   China      (1),
               Swaziland      (1),
                  Taiwan      (1),
                  Turkey      (1),
              Uzbekistan      (1),
                 Hungary      (1),
                Holy See      (1),
                   Ghana      (1).
                   
The geographic distribution of the participants indicates that our methods of distributing the survey were obviously insufficient at reaching researches in Asia, Africa and eastern Europe. Compared to the \href{http://www.iau.org/administration/membership/individual/distribution/}{IAU membership}, the USA is over-represented by a factor of 2.1, Germany by a factor of 2.7, and China is under-represented by a factor of 60. Major contributors to this imbalance are the language of the survey (English) and the method of distribution (social media and friends-of-friends networks). Hence, any conclusions we make will be only applicable to the researchers working in the countries which are predominantly represented.
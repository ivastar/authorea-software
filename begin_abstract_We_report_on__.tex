\begin{abstract}
We report on an informal survey of the software education and use in the worldwide astronomical community. The survey was carried out between December 2014 and February 2015, collecting the responses of 1142 astronomers, spanning all career levels. We find that all participants use software in their research. The vast majority of participants, 90\%, write at least some of their own software, while the remaining predominantly use software written by others. Even though writing software is so wide-spread among the survey participants, only 8\% of them report that they have received substantial training in software development. Another 49\% of the participants have received a little training. The remaining 43\% have received no training. The astronomer software stack is fairly narrow. Only 10 software tools are used by more than 10\% of the survey participants. These are (from most popular to least popular): Python, shell scripting, IDL, C/C++, Fortran, IRAF, spreadsheets, HTML/CSS, SQL and Supermongo. Across all participants the most common programing language is Python (67±2\%), followed by IDL (44±2\%), C/C++ (37±2\%) and Fortran (28±2\%). Shell scripting is the second most popular tool for astronomers (47±2\%). The software system IRAF (Image Reduction and Analysis Facility) is used frequently by 24±1\% and survey participants. We show that all trends are largely independent of career stage, area of research and geographic location.
\end{abstract}
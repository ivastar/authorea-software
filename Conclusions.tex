
\section{Conclusions}
\label{sec:conc}

%http://arxiv.org/abs/1212.1915
%Further, as we push for more researchers to publish their software in order to make research more reproducible and reuse more software, we need realize that one of the most common reasons for not sharing software is shame. Before we force researchers to make code public, we need to provide them with resources to learn how to write good code. 

Based on the responses summarized in Section \ref{sec:res}, we come to the following conclusions:



\begin{itemize}
\item Unsurprisingly, all 1142 survey participants say that they do use software in their research (Figure 1). The unanimous answer to this question underscores the importance of understanding how astronomers use software (i.e., the purpose of this survey).
\item $89\%$ of astronomers across all demographics write their own software (Figure \ref{fig:write1}).  However, only 58\% of those who write software (Figure \ref{fig:crosstrain}) are trained in software.  Moreover, only 8\% self-report as having better than ``a little'' training.  From this we conclude that 42\% of astronomers have no training for a key element of their work, and 92 \% have at most ``a little'' training.  
\item Python is the dominant language among our respondents.  Surprisingly, this is mostly independent of career stage (Figure \ref{fig:stack1}).  While a commonly-expressed mindset, reflected in some of the comments, is that graduate students are more likely to know the newer languages, it appears that this is not true, at least in our survey. 
\item Astronomers have a fairly narrow software stack, with only 10 tools used by more than 10\% or respondents. Theorists tend to have a ``narrow stack'' relative to other fields (Figure \ref{fig:stack2}) as do graduate students relative to more senior researchers. Independent of career level and field, Python and shell scripting are are most popular tools for astronomers. These results show that training efforts can have a significant impact even if they only focus on a limited number of software tools. The rankings we produce can be helpful in choosing training topics that would be most useful for the broadest group of participants. 
%\item There are no strong trends we see with nationality, although there is a hint that IDL and IRAF are more popular in the US, and C is particularly popular in Germany (Figure \ref{stack3}).
\end{itemize}

We caution that these results are tentative in part because our sampling methodology was not robust.  If nothing else, we hope this survey will prompt a more formal study of software use in astronomy to better understand how we should use the limited resources of our community to improve software training and software use.

%Future work: would be useful to also collect gender information. The UK study claims that men are more likely to develop their own software. 
\section{Comparison to SSI Survey}

90\% of astronomers write some of software (93\% of UK astronomers). This is much larger than the SSI survey which finds that 56\% of researchers write some of their software, i.e., astronomy depends much more on software than other sciences.

In the SSI survey, 55\% of respondents say they have received some training in software development, with 40\% indicating that the training was a formal course and 15\% indicating self-directed study. Our categories are not identical, but we find that a similar fraction of astronomers - 57\% (53\% of UK astronomers) say that they have received some form of training. However, only 8\% say that the training was substantial, while 49\% say they received a little training. The decision what constitutes a lot and a little was left to the participants, but those who chose to expand on their decision indicated that a lot corresponded to a formal class while a little corresponded to using on-line materials such as Software Carpentry and Code Academy.

Worryingly, 42\% of respondents who write their own software have received no training. This fraction is twice as large as the one in the SSI survey (21\%). That is 2 out of 5 astronomers! 


\section{Conclusions}


%http://arxiv.org/abs/1212.1915
%Further, as we push for more researchers to publish their software in order to make research more reproducible and reuse more software, we need realize that one of the most common reasons for not sharing software is shame. Before we force researchers to make code public, we need to provide them with resources to learn how to write good code. 

Tetative conclusions
\begin{itemize}
\item Most astronomers write their own software but are not trained to do it.
\item Python is the dominant language now, true roughly independent of age
\item Anything interesting about the Venn diagram, esp. IDL vs. Python
\item Theorists have a "narrow stack", also Grads but less so.  But the order pretty much stays the same.
\item therefore, just teach them those things and it should be good
\item Not much variation internationally, but (IDL and IRAF?) significantly more in the US vs. not, and Germans like C and David Hasslehoff
\end{itemize}

Future work: would be useful to also collect gender information. The UK study claims that men are more likely to develop their own software. 
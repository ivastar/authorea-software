
\section{Conclusions}
\label{sec:conc}

%http://arxiv.org/abs/1212.1915
%Further, as we push for more researchers to publish their software in order to make research more reproducible and reuse more software, we need realize that one of the most common reasons for not sharing software is shame. Before we force researchers to make code public, we need to provide them with resources to learn how to write good code. 

Based on the responses summarized in Section \ref{sec:res}, we come to the following conclusions:



\begin{itemize}
\item Unsurprisingly, all 1142 survey participants say that they do use software in their research (Figure 1). The unanimous answer to this question underscores the importance of understanding how astronomers use software (i.e., the purpose of this survey).
\item Most astronomers write their own software but are not trained to do it.
\item Python is the dominant language now, true roughly independent of age
\item Anything interesting about the Venn diagram, esp. IDL vs. Python
\item Theorists have a "narrow stack", also Grads but less so.  But the order pretty much stays the same.
\item therefore, just teach them those things and it should be good
\item Not much variation internationally, but (IDL and IRAF?) significantly more in the US vs. not, and Germans like C and David Hasslehoff
\end{itemize}

Future work: would be useful to also collect gender information. The UK study claims that men are more likely to develop their own software. 
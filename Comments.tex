\section{Comments}
\label{sec:comments}

We allowed participants to leave comments at the end of the survey to further get a sense of the attitude to this topics. We thank the participants for leaving so many thoughtful comments! In order to increase the anonymity of comments we detached them along with the career stage from the remainder of the results and randomized their order. We further removed e-mails, names and other identifying information. If anyone who took the survey would prefer that their comments are not included in the data, we kindly ask them to contact us and we would be happy to remove the information from the public dataset. 

We find that there are three recurring topics in the comments. 

The first common comments topic is the switch from IDL to Python. Many users comment that they would like to or are planning to make the switch between IDL and Python, frequently because of licensing issues and costs. We find it particularly striking that several senior astronomers commented that they are learning Python to be more helpful to their students.

The second common comments topic is the desire for more opportunities to improve software development skills. Many participants voiced interest in attending software development classes for astronomers. Several suggested that classes in programming and statistics should be an integral part of the undergraduate and/or graduate curriculum in astronomy. Patricipants suggested (and the authors agree) that such training not only aids in better research efficiency, but also can make code more readable and reusable. 

The third recurring topic in the comments is the lack of career opportunities for astronomers who write software. Suggestions were made that both more recognition should be given to those who spend most of their time developing important tools for the astronomical community and that such efforts should be appreciated, recognized and funded. 

We cite some of the representative comments on each topic in the Appendix.

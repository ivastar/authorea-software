\section{Comments}
\label{sec:comments}

We allowed participants to leave comments at the end of the survey to further get a sense of the attitude to this topics. We thank so many of the participants for leaving thoughtful comments. We find that there are three recurring topics in the comments. The first such topic is the switch from IDL to Python. 

\begin{itemize}
\item{I recently switched from using IDL as my primary programming language to using python. (Graduate student)}
\item{Mainly IDL user who wants to switch to python as it is more open source. (Graduate student)}
\item{I learned IDL as an undergrad and continue to mostly code in it as a grad student. However I've been learning Python lately and plan to mostly switch over within the next year or so. (Graduate student)}
\item{I plan to learn Python but haven't yet worked with it. (Graduate student)}
\item{At this stage I see Python as the future and am rapidly moving away from IDL. (Graduate studen)}
\item{I learned IDL as an undergrad (class of 2004) and used it nearly exclusively [...] until about two years ago. Over the last two years I've been slowly switching to Python [...]
(Postdoc)}
\item{I've only recently started working in IDL and Python.  I expect to do quite a bit of development in Python from now on. (Postdoc)}
\item{I want to learn python and R as soon as posible (Postdoc)}
\item{While I haven't learned it yet many of my colleagues use python and I make all of my students learn that (instead of e.g. matlab). (Faculty)}
\item{I am telling all of my students to learn Python and through that I am also gaining proficiency in python.  This is different from what my advisor did.  He told me to use the language that he used so that he could help me debug it. (Faculty)}
\item{Moving to Python as IDL needs a licence... And I just like the language. (Faculty)}
\item{Currently I code in IDL but I am trying to switch to python. (Faculty)}
\item{I plan to switch from IDL to python over the next 2-3 years. (Faculty)}
\item{I'm intending to try out Python soon. (Faculty)}
\item{I find python in particular to be incredibly useful and have taken to using its object-oriented features much more over the years.  During the stage of my career when I might have learned IDL it was too expensive for me and we didn't have a site license.  I might use it if that had not been the case. (Faculty)}
\item{Taking a workshop on Python soon I can encourage and help my students learn a language that can be used outside of academia as well. (Faculty)}
\end{itemize}

The second topic is the desire for more oportunities to improve software development skills:
\begin{itemize}
\item{If there is any software development training program for astronomers I would love to attend. (Graduate student)}
\item{I think it should be strongly recommended that people going into Astronomy should take programming classes. In fact I would make it part of the required course work to get a B.S. in Physics or Astronomy.Most astronomers I know did not take any formal programming course we learned as we went along. Most of us write our own code and many do not use good coding practices making reading or adapting code from other Astronomers a lot more painful than it should be. (Graduate student)}
\item{Helps to include courses in computation and statistics in grad curriculum. (Graduate student)}
\item{Astro students should get more formal training in programming and software design (and it should happen at the undergraduate level whenever possible). (Graduate student)}
\item{I wish I could get more formal training on programming. I have the feeling that Astronomy and Physics department usually don't emphasize the importance of programming until people start doing researches. gs}
\item{The coding skills incoming graduate students possess seem to vary wildly but they are often dismal with respect to the level required to begin doing serious research right away. In my department there seems to be little motivation to rectify this with either: (1) requiring undergraduate CS preparation as a condition for admission; or (2) organizing a programming course for beginning graduate students. This situation seems if not unsustainable very much non-optimal for the cultivation of strong substantive independent research skills. I imagine many other departments are currently facing the same dilemma.}
\item{More formal training in software use/development would be wonderful. gs}
\item{Formal training on astronomical software packages as part of my astronomy undergraduate degree would have been very helpful for me.}
\item{We need to be teaching undergraduates and graduates good object-oriented design skills from day one. Software is more important now than it ever was and the "learn as you go" mentality causes a tremendous amount of wasted efforts as bad code has to be rewritten all too often with the side effect of having astronomers with career skills that aren't as well developed as they could have been should they choose to leave astronomy.Also we should make a concerted effort to rewrite some of our legacy tools (IDL libraries IRAF etc) in a language and style that is more easily and cleanly extended and maintained. Incentives to put in the time and money for these [initially] low impact projects are hard to come by though.}

\end{itemize}

The third topic is the lack of career oportunities for people who write software.

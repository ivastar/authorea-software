\subsection{Do Astronomers Use Their Own Software?}

We ask the survey participants to best describe the authorship of the software that they use. The goal of this question is to find what is the predominant practice in the community: do most of us use "black box" software packages written by few or do most of us write custom software. We find that most often we do both (Figure \ref{fig:write1}, first panel): $57\%\pm2\%$ choose this option. One third of survey participants say that they mostly write their own software ($33\%\pm2\%$). And finally, only a small portion of survey participants predominantly use software written by others: $11\%\pm1\%$. Overall, 89\% of all participants write some of their own software.

In Figure \ref{fig:write1} we also explore the breakdown of the answers as a function of career stage. The answers vary slightly between the three groups. One curious result is that the answers of the "early" and "late" career stages closely resemble each other. The reason for such a trend may be that students follow their advisors' recommendations on software practices. In both of these groups, $\sim30\%(\pm3\%$) predominantly write their own software, while $\sim11\%(\pm2\%$) mostly use software written by others. In contrast, a larger portion of postdocs write their own software: $39\%\pm3\%$. 

We further consider the breakdown of answers as a function of research area in Figure \ref{fig:write2}. The groups are not fully independent because participants were allowed to choose more than one research area. In all groups the largest portion of astronomers, $\ge50\%$, use both software written by others and write their own. Researchers working in theory and instrumentation are more likely to primarily depend on their own software ($42\pm4\%$ and $38\pm5\%$, respectively) than planetary and observational astronomers ($32\pm7\%$ and $29\pm2\%$, respectively). The latter two groups are also more likely to use software written by others, 17$\pm5\%$ and 12$\pm1\%$ for planetary and observational researchers, respectively, versus 6$\pm1\%$ and 5$\pm2\%$ for theory and instrumentation.

Finally, we break down the answers by country of the researchers' home institution as shown in Figure \ref{fig:write3}. These and the following plots by country are more difficult to interpret because they only contain information about the researchers' current institutions rather than their institutional history. All countries show similar trends with $\ge50\%$ of astronomers choosing the "Both" option. At the extremes, the survey respondents from Germany are most likely to write their own software, $38\pm5\%$, and the respondents from the UK are least likely to use software written by others, $6\pm3\%$.

In conclusion, the majority of astronomers, $\sim90\%$ write at least some of their own software, across all demographics explored in our survey and a third of survey participants predominantly rely on their own software.  
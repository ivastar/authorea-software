\subsection{Are We Trained?}

Considering that, across all demographics, $\sim90\%$ of astronomers are involved in writing software (\S \ref{ssec:own}), it is important to asses the level of training we receive. We allowed participants to choose from one of three levels of training in software development: "A little", "A lot" and "None". The survey questions did not give guidelines on how to interpret the first two categories. Rather we left it to the survey participants to decide whether they thought their training was substantial or not. This section breaks down the answers into different demographics. 

The first panel in Figure \ref{fig:train1} shows the answers from all participants. Overall, $8\pm1\%$ of survey participants have received substantial training, $49\pm2\%$ have received a little training and $43\pm2\%$ have received no training. Altogether, $57\pm2\%$ of survey participants have received some training in software development. Across all career levels, only $\sim8\%$ of astronomers have received significant training. Facutly and scientists are slightly more likely to have received no training at $50\pm4\%$ versus $40\pm3\%$ for the more junior groups. Postdocs and graduate students are slightly more likely to have received some training at $53\pm4\%$, relative to faculty and scientists ($42\pm3\%$).

In Figure \ref{fig:crosstrain} we specifically focus on the training of survey participants who, in the previous questions, said that they primarily write their own software ("My own" option, $\sim33\%$ of the sample).  Overall, $40\pm3\%$ of those who write software have received no training and $89\pm5\%$ have received at best a little bit of training. This is of particular importance because it implies that many astronomers have little to no training in an activity that is a major part of their research work, despite the fact that they nearly always have many years of post-secondary education during which they could have received such training.

In Figure \ref{fig:train2} we show the breakdown of answers as a across of research areas. The trends remain the same across all fields.
The breakdown by country (Figure \ref{fig:train3}) shows that the results are similar internationally. The fraction of astronomers with significant training is largely independent of geography. Some geographical variations exist in the fraction of participants who have at least a little training: the USA has the largest fraction with training: $55\pm3\%$, while Australia has the smallest with $35\pm7\%$. Based on these results, we speculate that opportunities to receive at least a little bit of training are more available at US institutions or that more US researches seek out such opportunities. 

In conclusion, across all career levels, research areas and countries, only a small fraction of astronomy researchers receive significant training in software development. The lack of a strong trend with career level may indicate that significant training only occurs at the undergraduate level (and some participants left comments to that effect). While graduate students are more likely to have had a little training, it seems that few graduate programs offer and/or require CS courses. Overall, $\sim90\%$ of the survey participants have received only a little bit of training at best, despite \emph{all} being software users, and most being writers of their own software. 



    
    
    
  
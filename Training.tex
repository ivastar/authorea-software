\subsection{Are We Trained?}

Considering that across all demographics $\sim90\%$ of astronomers are involved in writing software to some degree, it is important to asses the level of training we receive. We allowed participants to choose from one of three levels of training in software development: "A little", "A lot" and "None". The survey questions did not give guidelines on how to interpret the first two categories. Rather we left it to the survey participants to decide whether they thought their training was substantial or not. This section breaks down the answers in different demographics. 

The first panel in Figure \ref{fig:train1} shows the answers from all participants. Overall $8\%\pm1\%$ of survey participants have received substantial training, $49\%\pm2\%$ have received a little training and $43\%\pm2\%$ have received no training. Altogether, 57\% of survey participants have received some training in software development. Across all career levels, only $\sim8\%$ of astronomers have received significant training. Facutly and scientists are slightly more likely to have received no training at $50\pm4\%$ versus $\sim40\pm3\%$ for the more junior groups. Postdocs and graduate students are slightly more likely to have received some training at $\sim53\pm4\%$, relative to faculty and scientists ($42\pm3\%$).

In Figure \ref{fig:crosstrain} we specifically focus on the training of survey participants who, in the previous questions, said that they do write some of their software ("My own" & "Both" options). This represents $\sim90\%$ of the sample. The results are indistinguishable from the full sample. Overall, $42\%\pm2\%$ of those who write software have received no training and $92\%\pm3\%$ have received at best a little bit of training.

In Figure \ref{fig:train2} we show the breakdown of answers as a function of research area. Across all fields the trends remain the same. The planetary science results seem slightly different, however the sample is small and they are consistent with the other sub-fields within the 1$\sigma$ errors. 

The break down by country (Figure \ref{fig:train3}) shows that the trends are similar internationally. The fraction of astronomers with significant training is largely independent of geography. Some geographical variations exist in the fraction of participants who have a little training: the USA has the largest fraction with $55\%\pm3\%$ while Australia has the smallest with $35\%\pm7\%$. The corresponding fractions of researchers with no training are reversed. These results may indicate (pure speculation here) that opportunities to receive at least a little bit of training are more available at US institutions or that more US researches seek out such opportunities. 

In conclusion, across all career levels, research areas and countries, only a small fraction of astronomy researches have received significant training in software development. The lack of a strong trend with career level may indicate that significant training only occurs at the undergraduate level (and some participants left comments to that effect). While graduate students are more likely to have had a little training, few graduate programs offer and/or require CS courses.   Overall, $\sim90\%$ of the survey participants have received only a little bit of training at best. 



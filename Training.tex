\subsection{Are We Trained?}

Considering that across all demographics $\sim90\%$ of astronomers are involved in writing software to some degree, it is important to asses the level of training we receive. We allowed participants to choose from one of three levels of training in software development: "A little", "A lot" and "None". This section breaks down the answers in different demographics. 

The first panel in Figure \ref{fig:train1} shows the answers from all participants. Both overall and for the individual  $8\%\pm1\%$ of survey participants have received substantial training, $49\%\pm2\%$ have received a little training and $43\%\pm\2$ have received no training. Altogether, 57\% of survey participants have received some training in software development. Across all career levels, only $\sim8\%$ of astronomers have received significant training. Facutly and scientists are slightly more likely to have received no training at $50\pm4\%$ versus $\sim40\pm3\%$ for the more junior groups. Postdocs and graduate students are slightly more likely to have received some training at $\sim53\pm4\%$, relative to faculty and scientists ($42\pm3\%$).

Figure \ref{fig:crosstrain} specifically focuses on the training of survey participants who, in the previous questions, said that they do write some of their software ("My own" & "Both" options). This represents $\sim90\%$ of the sample. BLAH< BLAH< BLAH.

In Figure \ref{fig:train2} we show the breakdown of answers as a function of research area. Across all fields the trends remain the same. The planetary science results seem slightly different, however the sample is small and they are consistent with the other sub-fields within the 1$\sigma$ errors. The break down by country (Figure )

